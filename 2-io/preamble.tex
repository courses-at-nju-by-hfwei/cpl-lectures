% preamble.tex

%%%%%%%%%%%%%%%%%%%%%%%%%%%%%%%%%%%%%%%%%%%%%%%%%%%%
\usepackage{lmodern}
\usepackage{xeCJK}

\usetheme{CambridgeUS} % try Madrid, Pittsburgh
\usecolortheme{beaver}
\usefonttheme[]{serif} % try "professionalfonts"

% to download and install the Merriweather font:
% https://fonts.google.com/specimen/Merriweather
\setmainfont{Merriweather} % use xelatex
% \renewcommand{\today}{\number\year 年\number\month 月\number\day 日}

\setbeamertemplate{itemize items}[default]
\setbeamertemplate{enumerate items}[default]

\usepackage{amsmath, amsfonts, latexsym, mathtools, centernot, wasysym, pifont}
\usepackage{dingbat}%\eye and \leftpointright
\newcommand{\N}{\mathbb{N}}
\newcommand{\Z}{\mathbb{Z}}
\newcommand{\R}{\mathbb{R}}
\newcommand{\set}[1]{\{#1\}}
\newcommand{\bset}[1]{\big\{#1\big\}}
\newcommand{\Bset}[1]{\Big\{#1\Big\}}

\usepackage{ulem}
\usepackage{savesym}
\savesymbol{checkmark} % checkmark defined in dingbat
\usepackage{dingbat}
\newcommand{\cmark}{\green{\ding{51}}}
\newcommand{\xmark}{\red{\ding{55}}}

\usepackage{algorithm}
\usepackage[noend]{algpseudocode}
\newcommand{\hStatex}[0]{\vspace{5pt}}

\usepackage{multirow}
\newcommand{\incell}[2]{\begin{tabular}[c]{@{}c@{}}#1\\ #2\end{tabular}}

\usepackage{xcolor}
% color
\newcommand{\red}[1]{\textcolor{red}{#1}}
\newcommand{\redoverlay}[2]{\textcolor<#2>{red}{#1}}
\newcommand{\gray}[1]{\textcolor{gray}{#1}}
\newcommand{\green}[1]{\textcolor{green}{#1}}
\newcommand{\blue}[1]{\textcolor{blue}{#1}}
\newcommand{\blueoverlay}[2]{\textcolor<#2>{blue}{#1}}
\newcommand{\teal}[1]{\textcolor{teal}{#1}}
\newcommand{\purple}[1]{\textcolor{purple}{#1}}
\newcommand{\cyan}[1]{\textcolor{cyan}{#1}}
\newcommand{\brown}[1]{\textcolor{brown}{#1}}
\newcommand{\yellow}[1]{\textcolor{yellow}{#1}}
\newcommand{\violet}[1]{\textcolor{violet}{#1}}
% color box
\newcommand{\rbox}[1]{\red{\boxed{#1}}}
\newcommand{\gbox}[1]{\green{\boxed{#1}}}
\newcommand{\bbox}[1]{\blue{\boxed{#1}}}
\newcommand{\pbox}[1]{\purple{\boxed{#1}}}

\usepackage{minted} % code formatting and highlighting
\usemintedstyle{vim}
%%%%%%%%%%%%%%%%%%%%%%%%%%%%%%%%%%%%%%%%%%%%%%%%%%%%%%%%%%%%%%
% for fig without caption: #1: width/size; #2: fig file
\newcommand{\fig}[2]{
  \begin{figure}[htp]
    \centering
    \includegraphics[#1]{#2}
  \end{figure}
}

% for fig with caption: #1: width/size; #2: fig file; #3: caption
\newcommand{\figcap}[3]{
  \begin{figure}[htp]
    \centering
    \includegraphics[#1]{#2}
    \caption{#3}
  \end{figure}
}

% for fig with caption and label: #1: width/size; #2: fig file; #3: caption; #4: label
\newcommand{\figcaplbl}[4]{
  \begin{figure}[htp]
    \centering
    \includegraphics[#1]{#2}
    \caption{#3}
    \label{#4}
  \end{figure}
}
%%%%%%%%%%%%%%%%%%%%%%%%%%%%%%%%%%%%%%%%%%%%%%%%%%%%
\newcommand{\thankyou}{
\begin{frame}[noframenumbering]{}
  \fig{width = 0.50\textwidth}{figs/thankyou.png}
\end{frame}
}
%%%%%%%%%%%%%%%%%%%%%%%%%%%%%%%%%%%%%%%%%%%%%%%%%%%%
\usepackage{tikz}
\usepackage{tikz-qtree}
\usetikzlibrary{calc, shapes}
%%%%%%%%%%%%%%%%%%%%%%%%%%%%%%%%%%%%%%%%%%%%%%%%%%%%
\usepackage[tikz]{bclogo}
\usepackage[framemethod=tikz]{mdframed}

\tikzset{errorsymbol/.style = {scale = 2}}

\tikzset{lampsymbol/.style={
  scale = 2, overlay}}

\newmdenv[hidealllines = true, backgroundcolor = red!20,
  frametitle = {\textsc{Pitfalls}},
  frametitlefont = \color{red}\serif,
  skipabove = \topsep, skipbelow = \topsep, nobreak,
  leftmargin=.2cm, rightmargin=.2cm,
  singleextra={\path let \p1=(P), \p2=(O) in ($(\x2,0) + 0.5*(2,\y1)$)
    node (eye) [errorsymbol] {\eye};},
  innerleftmargin = 1.8cm,
]{error}

\newmdenv[nobreak,
  middlelinewidth=.8pt,
  frametitlefont=\bfseries,
  leftmargin=.2cm, rightmargin=.2cm,
  innerleftmargin=2cm,
  skipabove=\topsep,skipbelow=\topsep,
  singleextra={\path let \p1=(P), \p2=(O) in ($(\x2,0)+0.5*(2,\y1)$)
    node (lamp) [lampsymbol] {\leftpointright};},
]{lamp}
%%%%%%%%%%%%%%%%%%%%%%%%%%%%%%%%%%%%%%%%%%%%%%%%%%%%
% preamble.tex

\usepackage{lmodern}
\usepackage{xeCJK}
% \usepackage{verbatim}
% \usepackage{fancyvrb} % to replace 'verbatim'

\usetheme{CambridgeUS} % try Madrid, Pittsburgh
\usecolortheme{beaver}
\usefonttheme[]{serif} % try "professionalfonts"

% to download and install the Merriweather font:
% https://fonts.google.com/specimen/Merriweather
% \setmainfont{Merriweather} % use xelatex

\setbeamertemplate{itemize items}[default]
\setbeamertemplate{enumerate items}[default]

\usepackage{amsmath, amsfonts, latexsym, mathtools, centernot}
\newcommand{\set}[1]{\{#1\}}
\newcommand{\bset}[1]{\big\{#1\big\}}
\newcommand{\Bset}[1]{\Big\{#1\Big\}}
\newcommand{\ps}[1]{\mathcal{P}(#1)}
\newcommand{\card}[1]{\Big\lvert #1 \Big\rvert}

\DeclareMathOperator*{\argmin}{\arg\!\min}
% nature deduction: #1: premise; #2: conclusion; #3: name
\newcommand{\nd}[3]{\frac{\quad#1\quad}{\quad#2\quad} \quad (#3)}
\newcommand{\ndnoname}[2]{\frac{\quad#1\quad}{\quad#2\quad}}

% Begin: use JetBrains Mono %
% see https://tex.stackexchange.com/a/548156/23098
\usepackage{fontspec}
% \setmonofont{JetBrains Mono}[
%     Contextuals = Alternate,
%     Ligatures = TeX,
% ]

\usepackage{listings}
\lstset{
  basicstyle = \ttfamily,
  columns = flexible,
}
\makeatletter
\renewcommand*\verbatim@nolig@list{}
\makeatother
% End: use JetBrains Mono %

\definecolor{bgcolor}{rgb}{0.95,0.95,0.92}

\newcommand{\incell}[2]{\begin{tabular}[c]{@{}c@{}}#1\\ #2\end{tabular}}

\usepackage{algorithm}
\usepackage[noend]{algpseudocode}
\newcommand{\hStatex}[0]{\vspace{5pt}}

\usepackage{multirow}

\renewcommand{\today}{\number\year 年\number\month 月\number\day 日}

% \usepackage{hyperref}
% \hypersetup{
%   colorlinks = flase,
%   linkcolor = blue,
%   filecolor = magenta,
%   urlcolor = teal,
% }

% colors
\newcommand{\red}[1]{\textcolor{red}{#1}}
\newcommand{\redoverlay}[2]{\textcolor<#2>{red}{#1}}
\newcommand{\gray}[1]{\textcolor{gray}{#1}}
\newcommand{\green}[1]{\textcolor{green}{#1}}
\newcommand{\blue}[1]{\textcolor{blue}{#1}}
\newcommand{\blueoverlay}[2]{\textcolor<#2>{blue}{#1}}
\newcommand{\teal}[1]{\textcolor{teal}{#1}}
\newcommand{\purple}[1]{\textcolor{purple}{#1}}
\newcommand{\cyan}[1]{\textcolor{cyan}{#1}}
\newcommand{\brown}[1]{\textcolor{brown}{#1}}
\newcommand{\yellow}[1]{\textcolor{yellow}{#1}}
\newcommand{\violet}[1]{\textcolor{violet}{#1}}

% color box
\newcommand{\rbox}[1]{\red{\boxed{#1}}}
\newcommand{\gbox}[1]{\green{\boxed{#1}}}
\newcommand{\bbox}[1]{\blue{\boxed{#1}}}
\newcommand{\pbox}[1]{\purple{\boxed{#1}}}

\usepackage{listings}
\usepackage{xcolor}

\definecolor{codegreen}{rgb}{0,0.6,0}
\definecolor{codegray}{rgb}{0.5,0.5,0.5}
\definecolor{codepurple}{rgb}{0.58,0,0.82}
\definecolor{backcolour}{rgb}{0.95,0.95,0.92}

\lstdefinestyle{compstyle}{
  backgroundcolor=\color{backcolour},
  commentstyle=\color{codegreen},
  keywordstyle=\color{magenta},
  numberstyle=\tiny\color{codegray},
  stringstyle=\color{codepurple},
  basicstyle=\ttfamily\footnotesize,
  breakatwhitespace=false,
  breaklines=true,
  captionpos=b,
  keepspaces=true,
  numbers=left,
  numbersep=5pt,
  showspaces=false,
  showstringspaces=false,
  showtabs=false,
  tabsize=2
}
\lstset{style=compstyle}

\usepackage{pifont}
\usepackage{wasysym}
\usepackage{ulem}

\usepackage{savesym}
\savesymbol{checkmark} % checkmark defined in dingbat
\usepackage{dingbat}

\newcommand{\cmark}{\green{\ding{51}}}
\newcommand{\xmark}{\red{\ding{55}}}

% see https://tex.stackexchange.com/a/109906/23098
\usepackage{empheq}
\newcommand*\widefbox[1]{\fbox{\hspace{2em}#1\hspace{2em}}}
%%%%%%%%%%%%%%%%%%%%%%%%%%%%%%%%%%%%%%%%%%%%%%%%%%%%%%%%%%%%%%
% for fig without caption: #1: width/size; #2: fig file
\newcommand{\fig}[2]{
  \begin{figure}[htp]
    \centering
    \includegraphics[#1]{#2}
  \end{figure}
}

% for fig with caption: #1: width/size; #2: fig file; #3: caption
\newcommand{\figcap}[3]{
  \begin{figure}[htp]
    \centering
    \includegraphics[#1]{#2}
    \caption{#3}
  \end{figure}
}

\newcommand{\N}{\mathbb{N}}
\newcommand{\Q}{\mathbb{Q}}
\newcommand{\Z}{\mathbb{Z}}
\newcommand{\R}{\mathbb{R}}
\renewcommand{\U}{\mathcal{U}}

\usepackage{tabu}

\DeclareRobustCommand{\stirling}{\genfrac\{\}{0pt}{}}
\newcommand{\cnum}[1]{\blue{#1}} % \textcircled{1}

\newcommand{\thankyou}{
\begin{frame}[noframenumbering]{}
  \fig{width = 0.50\textwidth}{figs/thankyou.png}
\end{frame}
}

\usepackage{tikz}
\usepackage{tikz-qtree}